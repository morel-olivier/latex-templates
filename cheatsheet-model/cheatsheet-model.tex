%%%%%%%%%%%%%%%%%%%%%%%%%%%%%%%%%%%%%%%%%%%%%%%%%%%%%%%%%%%%%%%%%%%%%%
% cheatsheet-model.tex
% V1
%
% Original Source: Dave Richeson (divisbyzero.com), Dickinson College
% Modified By: Paul Gessler, Overleaf (overleaf.com)
% Modified By: Olivier Morel
% 
% Feel free to distribute this example, but please keep the referral
% to divisbyzero.com
%%%%%%%%%%%%%%%%%%%%%%%%%%%%%%%%%%%%%%%%%%%%%%%%%%%%%%%%%%%%%%%%%%%%%%


\documentclass[12pt,a4paper, landscape]{article}
\usepackage[french]{babel}
\usepackage[T1]{fontenc}
\usepackage[utf8]{inputenc}
\usepackage{amssymb,amsmath,amsthm,amsfonts}
\usepackage{multicol,multirow}
\usepackage{spverbatim}
\usepackage{graphicx}
\usepackage{ifthen}
\usepackage[landscape]{geometry}
\usepackage[colorlinks=true,urlcolor=olgreen]{hyperref}
\usepackage{booktabs}
\usepackage{microtype}
\usepackage{tabularx}
\usepackage{siunitx}
\usepackage{graphicx}
\usepackage{float}
\usepackage{xcolor}

\sisetup{
	locale = FR,
	round-mode = figures,
	round-precision = 4,				% 4 digits in total
	scientific-notation = engineering,	% exponnant by multiple of 3
	per-mode            = symbol,
	per-symbol          = /,
	exponent-to-prefix  = true,
	output-decimal-marker = {.},		% decimal marker will always be printed as a '.'
	output-complex-root = j,			% imaginary unit will always be 'j'
	exponent-product = \cdot,			% exponant product will be a dot
	detect-all							% garde la même fonte que le reste du texte
}

% réduit la taille des marges
\ifthenelse{\lengthtest { \paperwidth = 11in}}
    { \geometry{margin=0.4in} }
	{\ifthenelse{ \lengthtest{ \paperwidth = 297mm}}
		{\geometry{top=1cm,left=1cm,right=1cm,bottom=1cm} }
		{\geometry{top=1cm,left=1cm,right=1cm,bottom=1cm} }
	}
\pagestyle{empty}	% supprime les numéros de pages
\makeatletter

% adaptation des titres pour être plus petits
\renewcommand{\section}{\@startsection{section}{1}{0mm}%
                                {-1ex plus -.5ex minus -.2ex}%
                                {0.5ex plus .2ex}%x
                                {\sffamily\large\color{blue}}}
\renewcommand{\subsection}{\@startsection{subsection}{2}{0mm}%
                                {-1explus -.5ex minus -.2ex}%
                                {0.5ex plus .2ex}%
                                {\sffamily\normalsize\itshape\color{cyan}}}
\renewcommand{\subsubsection}{\@startsection{subsubsection}{3}{0mm}%
                                {-1ex plus -.5ex minus -.2ex}%
                                {1ex plus .2ex}%
                                {\normalfont\small\itshape}}
\makeatother
\setcounter{secnumdepth}{0}	% suprime les numéros de titres
\setlength{\parindent}{0pt}
\setlength{\parskip}{0pt plus 0.5ex}
% -----------------------------------------------------------------------

\begin{document}
\footnotesize	% réduit la taille de police
%\raggedright

% titre
\begin{center}
  {\huge\sffamily\bfseries TITRE} \\
\end{center}

\setlength{\premulticols}{0pt}
\setlength{\postmulticols}{0pt}
\setlength{\multicolsep}{1pt}
\setlength{\columnsep}{1.8em}
\begin{multicols*}{3}

%--------------------------------------------------------------------------------------------------------------

\section{Energie et puissance}

\subsection{Puissance moyenne}

\begin{tabularx}{\columnwidth}{Xr}
	$P(t_1, t_2) = \cfrac{1}{t_2 - t_1} \int_{t_1}^{t_2}p(t)\, dt$	& \si{\watt}	\\
\end{tabularx}

\subsection{Signaux à énergie finie}
Les signaux périodiques ont forcément une énergie \emph{infinie}

\section{Périodicité discréte}
Lorsqu'un signal continu $x(t)$ de pulsation $\omega = 2 \pi f$ est échantilloné dans un signal $x[n]$ à une pulsation $\Omega = 2 \pi F_s$, on ne peut pas forcément détecter sa pulsation d'origine.

\[ \cfrac{2\pi k}{N} = \Omega_0\]

\section{Opération élementaire sur les signaux discret}
\begin{tabularx}{\columnwidth}{Xcr}
	décalage à gauche (le graph part à droite)	& $x[n-1]$	& $Z^{-1}$	\\
	décalage à droite (le graph part à gauche)	& $x[n+1]$	& $Z^{+1}$	\\
\end{tabularx}

\end{multicols*}
\end{document}
